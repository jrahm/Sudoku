
\documentclass{sig-alternate-05-2015}


\usepackage{svg}
\usepackage{pstricks}
\begin{document}

\title{Backtracking Sudoku}
\newcommand{\sep}{\vspace{1cm}}

\numberofauthors{1} %  in this sample file, there are a *total*
% of EIGHT authors. SIX appear on the 'first-page' (for formatting
% reasons) and the remaining two appear in the \additionalauthors section.
%
\author{
\alignauthor
Joshua Rahm\\
       \affaddr{University of Colorado}\\
       \email{joshua.rahm@colorado.edu}
}
\date{\today}
% Just remember to make sure that the TOTAL number of authors
% is the number that will appear on the first page PLUS the
% number that will appear in the \additionalauthors section.

\newcommand\citationneeded{${}^{\text{<citation\_needed>}}$}

\maketitle
\begin{abstract}
    This paper analyzes the runtime and the semantics of
    a backtracking Sudoku solver. To better undestand
    and analyze this algorithm, I have implemented
    a version of the backtracking sudoku solver in
    Haskell. This will be the base implementation from
    which the analysis will stem from.
\end{abstract}

\section{Introduction}

Sudoku is a very popular class of logic puzzles where the goal of the player is
to place numbers in cells in a grid adhering to certain constraints. The puzzle
consists of a grid, partially filled with numbers.  The grid is sized to be
$n^2\times n^2$ where $n$ is an integer. The grid is then divided up into $n$
$n\times n$ sub grids, as well as $n^2$ rows and columns. The goal of the
puzzle is to place numbers such that each row, column and sub-grid has every
number on the interval $[1, n^2]$.

For example, following is a valid solution for $n=3$ ($9\times 9$ grid)

\begin{figure}
    \begin{center}
        \sep
        \def\arraystretch{1.5}
        \begin{tabular}{| c c c | c c c | c c c |}
            \hline
            3&7&6&1&4&2&9&5&8 \\
            5&8&4&6&7&9&2&1&3 \\
            9&1&2&8&3&5&4&7&6 \\
            \hline
            6&9&1&2&5&4&3&8&7 \\
            8&4&5&3&6&7&1&9&2 \\
            2&3&7&9&1&8&5&6&4 \\
            \hline
            4&6&9&5&8&3&7&2&1 \\
            7&2&8&4&9&1&6&3&5 \\
            1&5&3&7&2&6&8&4&9 \\
            \hline
        \end{tabular}
        \sep
    \end{center}

    \caption{Example Complete Sudoku Board}
    \label{fig:example}
\end{figure}

As you can see in the example above, all boxes, rows and columns contain all
the numbers from 1 to 9.

\section{Algorithm}

The backtracking algorithm I implemented is fairly simple in concept. Sudoku is
a game where each row, column and block have a set of numbers from 1 to $n^2$
associated with that block. The first step in the algorithm is to initialize
the $SudokuPossibilities$, this is a data structure that keeps track of all
the unused numbers for each row, column and block.

The following pseudocode describes how the algroithm is implement.


\section{Analysis}

\subsection{Correctness}

The correctness of this Sudoku solving algorithm is in tow steps. % BAD!!!
First, I need to show that the solution the algorithm finds are actually
correct solutions. This means tow things. First, it means that the original
hints are in fact still in the solution! (The solver is not allowed to ignore
the hints and return just any valid board.) In addition, the board state must
be valid. This means that all rows, columns, and blocks must have unique
numbers from $0..n$.

This algorithm is designed such that doing any to the contrary is impossible.

\subsection{Run Time}

The general runtime for backtracking Sudoku is $O(N^H)$ where $H$ is the number
of hints given. \citationneeded.

\subsection{Empirical Results}

To gather information on the runtime based on the factors identified in the section
above. The results were generated as such; for $N=2,3,4$, and $H=0.2N,0.4N,0.6N,0.8N$.
Each configuration of the problem had $1000$ runs where each run was a randomly
generated Sudoku game for that configuration.

In the interest of time, the runtime was capped at 10 minutes, so this will have
skewed the results for the $N=4$ cases.

\begin{figure*}
    \begin{center}
        \resizebox{\textwidth}{!}{%
        \input{../runs/output.pdf_tex}}
        \caption{Measured Results}
        \label{fig:measured}
    \end{center}
\end{figure*}

Figure \ref{fig:measured} is an in-depth look at the runtime for different configurations of
Sudoku. The results clearly show a correlation between runtime and both factors. Each run
is displayed as a semi-transparent circle, this represents a histogram. Where the line is darker,
there are more runs that took that amount of time.


The lines indicate the mean and median of the runtimes for each $N$, showing the relationship
between $H$ and the runtime. The teal line indicates the median runtime for each $N$. This, shows
an super-exponential relationship between $N$ and the runtime.

Interestingly enough, there seems to be little consistency in the relationship
for $H$ and $T$. In the $N=2$ and $N=4$ cases, $H=0.4N$ is the worst case, whereas,
$H=0.2N$ is the worst case for $N=3$. This could be a result of the timeout not being
high enough. There is a possibility that the timeout has skewed these results and
the $N=4$ case does indeed look like the $N=3$ case, but considering the distributions
of the unskewed data, this is probably not the case.

\subsection{Space Usage}

\section{Pathological Cases}

\end{document}
